\sekce{Spuštění kontroly, tabulka změn, informační konzole}
Po spuštění aplikace se zobrazí hlavní okno.
Jelikož aplikace by měla běžet jako služba, okno neobsahuje tlačítko zavřít.
V~případě, že uživatel skutečné chce uzavřít aplikaci a tím i zabránit dalším automatickým kontrolám, je možné aplikaci ukončit z~menu kliknutím na \c|app| $\rightarrow$ \c|quit|.

Pro manuální spuštění kontroly je možné kliknout na tlačítko \c|Start checking|.
V~případě, že již kontrola běží, po kliknutí se ukončí a ihned začne znovu od začátku.
Po kliknutí na \c|Stop checking| se prohledávání neprodleně ukončí.

Při průchodu se nalezené změny ihned zobrazují do tabulky změn.
Tedy i v~případě, že je prohledávání ukončeno v~průběhu, doposud nalezené změny budou uloženy a zobrazeny.

Aktuální stav průchodu je vidět na stavovém baru  nad tlačítky.
V~případě, že bar má červenou barvu, alespoň jedna stránka nebyla při posledním průchodu úspěšně načtena.
Po kliknutí na zaškrtávací políčko \c|View more information| se zobrazí záznam o~provedených kontrolách a případně důvod jejich neúspěchu v~informační konzoli.
Všechny chybové hlášky jsou obsaženy na řádcích uvozených několika vykřičníky.
Nejčastěji se může uživatel setkat s~těmito chybami:\\
\c|Connection error - TIME OUT.| nastane v~případě, že načítání stránky bylo moc pomalé a tedy byl překročen časový limit na přístup na jednu stránku.\\ %isdo Opravit conection -> connection aplikaci
\c|Cookie error - cookie not available.| se zobrazí v~případě, že nastala chyba při načítání některé z~předcházejících stránek ze stejné cookie skupiny.\\
\c|Connection error - Network access is disabled.| znamená nedostupnost síťového připojení.\\
\c|Connection error - Host not found.| informuje o~nedostupnosti daného serveru. S~největší pravděpodobností se  jedná o~chybně zadanou stránku nebo byla přesunutá.
\obr{chyba-pristupu}{Výpis chyby v~informační konzoli.}

V~boxu vpravo od tlačítek je možno nastavit automatické spouštění kontrolování.
Do boxu stačí napsat počet minut mezi automatickými kontrolami.
Automatické kontroly lze vypnout napsáním \c|0| do políčka.
Datum a čas poslední úspěšné (tedy takové, ve které se načetly všechny stránky) kontroly je vidět mezi tlačítky a boxem automatické aktualizace.
V~tomto místě je také vidět čas příští plánované kontroly.

Informace o~poslední kontrole a periodě automatické kontroly si aplikace ukládá do souboru databáze v~pracovním adresáři aplikace.
Při spuštění si tento soubor načte (pokud existuje).
Nastavení automatické kontroly tedy vydrží i vypnutí a zapnutí aplikace.
V~případě, že kontrola měla proběhnout v~momentě, kdy byla aplikace vypnutá, proběhne neprodleně po jejím spuštění.

Když aplikace zaregistruje změnu stránky a není momentálně aktivní, bude se na změnu snažit upozornit.
Provedení této funkce je částečně závislé na platformě. 
Aplikace pošle požadavek na vyskočení do popředí, což ve většině případů znamená červené rozblikání dané aplikace případně ikony dané pracovní plochy.

Všechny nalezené změny se zobrazují v~tabulce změn umístěné v~horní části hlavního okna aplikace.
O~každé změně se zobrazí řádek obsahující jméno stránky, čas detekování změny a jméno souboru, v~němž je uložena aktuální verze.
V~případě, že uživatel chce data setřídit podle některé z~těchto položek, může tak učinit kliknutím na hlavičku daného sloupce tabulky.
Kliknutím na jméno souboru se ve výchozím prohlížeči otevře daná verze stránky.
Aby bylo zajištěno správné fungovaní zobrazení, jsou všechny relativní odkazy (v~rámci serveru) přepsány na absolutní.
Před všechny odkazy v~atributech \c|href| a \c|src|, které neobsahují absolutní cestu, je tedy doplněn název serveru a složky.
Díky tomu se při kontrole načítá pouze samostatná stránka, ale při zobrazení se načtou i obrázky, styly a další odkazované elementy.

V~případě, že už si uživatel danou změnu prohlédl, kliknutím na buňku v~sloupci \c|Delete| ji může z~tabulky změn odstranit.
Tímto odstraněním nedojde k~smazání záznamu o~změně ani odstranění souboru s~danou verzí stránky.
Změna se již nebude zobrazovat v~tabulce změn.
Jiným způsobem odstranění je vybrat jeden nebo několik řádků a pak kliknout na tlačítko \c|Hide| těsně pod tabulkou.

Vybráním řádků a kliknutím na tlačítko \c|Delete| dojde k~smazání vybraných záznamů změn včetně souborů obsahujících dané verze stránek.

Všechny záznamy o~změnách se ukládají do databáze a při spuštění aplikace se načtou všechny neskryté záznamy. 
