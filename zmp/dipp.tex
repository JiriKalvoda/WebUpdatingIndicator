\input{hlavicka.tex}
\begin{document}

\def\obrazek#1#2#3{
	\begin{figure}[tbhp]
  \centering
	{#1}
	\caption{{#2}}
  \label{fig:#3}
\end{figure}
	}
\def\obr#1#2{\obrazek{\includegraphics[width=9cm]{screenshots/#1.png}}{#2}{#1}}
\def\obrsize#1#2#3{\obrazek{\includegraphics[width=#3cm]{screenshots/#1.png}}{#2}{#1}}
\def\,{\penalty10000\hskip.25em}
\pagestyle{headings}

\cislovat{2}
\bakalarska


\titul{Vývoj aplikace na kontrolování změn na webových stránkách}{Jiří Kalvoda}{Mgr. Marek Blaha}{Blansko 2020}

\podekovani{Chtěl bych poděkovat Mgr. Marku Blahovi za odborné vedení práce, věcné připomínky a podání cenných rad.}

\prohlaseni{Prohlašuji, že jsem tuto práci vyřešil samostatně
s~použitím literatury, kterou uvádím v~seznamu}{V~Blansku dne \today}

\abstract{Kalvoda, J. Development an application for checking changes on web pages}
{This thesis describe developing and using an application monitoring changes on web pages.
This app is produced in c++ language using Qt library. Therefore, the app can be used on multiple platforms.
This app come under GNU LGPL licence and is available whit source code. This thesis contain description of its working and controlling, implementation and used software when it has been developed.}

\abstrakt{Kalvoda, J. \NAZEV}
{Tato závěrečná práce popisuje vývoj a použití aplikace na monitorování změn na webových stránkách.
Aplikace je vyvíjena v~jazyce c++ pomocí knihovny Qt. Díky tomu se jedná o~multiplatformní software.
Je dostupná včetně zdrojového kódu pod licencí GNU LGPL. Tato závěrečná práce obsahuje popis jejího fungování, implementace a použitého softwaru při jejím vývoji.}

\obsah

\input{uvodACil.tex}


\kapitola{Popis fungování a ovládání aplikace}
\sekce{Instalace}
\podsekce{Kompilace ze zdrojového kódu}
Způsobem, jak aplikaci nainstalovat na většině používaných operačních systémů, je kompilace ze zdrojového kódu.
Zdrojové kódy aktuální stabilní verze je možné stá\-hnout z~\url{gitlab.com/JiriKalvoda/webupdatingindicator/tree/master}.
V~pří\-padě, že má uživatel zájem o~aktuálně nejnovější funkce, je možné použít testovací verzi produktu dostupnou z~\url{gitlab.com/JiriKalvoda/webupdatingindicator/tree/Test}.
Soubory lze stáhnout pomocí webového rozhraní a nebo je lze naklonovat s~použitím gitu.
Aplikaci pak lze zkompilovat za použití Qt knihoven.
Nejsnazší způsob je využít aplikace Qt Creator. Pomocí ní stačí otevřít soubor \c|WebUpdatingIndicator.pro| a v~levém rohu aplikace kliknout na tlačítko pro kompilaci.
Tímto způsobem by měla vzniknout samostatně spustitelná aplikace, kterou stačí umístit do požadované složky a v~ní ji spouštět.
Podporovaná by měla být libovolná verze Qt větší než 5.4.
Pro vývoj se používá Qt 5.12.5.

V~případě, že uživatel nechce provádět kompilaci ze zdrojového kódu, pro základní architektury a operační systémy je možné využít již zkompilované varianty.
Ty jsou dostupné na adrese \url{gitlab.com/JiriKalvoda/webupdatingindicator-install/tree/master}
\podsekce{Linux}
Na operačních systémech postavených na jádře Linuxu stačí pouze stáhnout a rozzipovat složku s~programem do uživatelem zvoleného adresáře.
Pracovní adresář aplikace je pak ten, ze kterého se aplikace spouští (nemusí tedy být shodný s~adresářem, ve kterém je umístěna aplikace).
Pro jednodušší spouštění je dobré vytvořit bash script, který bude obsahovat přepnutí polohy do pracovního adresáře a spuštění aplikace.
%Příklad takového skriptu je uložen ve složce, ve které je umístěn program, pod názvem \c|run.sh|.
Pro možnost spouštění aplikace z~menu či pomocí přímého příkazu je možné tento skript umístit do adresáře \c|usr/bin|.

V~případě užívání správce oken i3 je vhodné nastavit, aby se okna porovnávání stránek zobrazovala jako plovoucí.
Toho lze docílit přidáním řádku \c|for_window [title="WebUpdatingIndicator compare"] floating enable| do konfiguračního souboru i3 umístěného v~\c|~/.config/i3/config|.
Pro snazší spouštění aplikace je také vhodné nadefinovat klávesovou zkratku.
Případně je možně vyhradit aplikaci speciální pracovní plochu a definovat její spuštění a přepnutí na danou plochu pomocí příkazů (nastaví spuštění na \c|$mod+Shift,| a zobrazení na \c|$mod+,|):\\
\begin{tabular}{l}
\c|bindsym $mod+Shift+comma workspace WUI;exec WebUpdatingIndicator.sh|\\
\c|bindsym $mod+comma workspace WUI|\\
\end{tabular}
%\obr{linux-i3-2}{Aplikace otevřená v~operačním systému Linux Mint 19 \protect\linebreak se správcem oken i3.}
%\def\obr#1\obrazek{}{#2}{#1}}
\begin{figure}[tbhp]
  \centering
	\includegraphics[width=9cm]{screenshots/linux-i3-2.png}
		\begin{minipage}{0.8\textwidth}
			\caption{Aplikace otevřená v~operačním systému Linux Mint 19 se správcem oken i3.}
		\end{minipage}
  \label{fig:linux-i3-2}
\end{figure}
\begin{figure}[tbhp]
  \centering
	\includegraphics[width=9cm]{screenshots/linux-cinnamon.png}
		\begin{minipage}{0.8\textwidth}
			\caption{Aplikace otevřená v~operačním systému Linux Mint 19 se správcem oken Cinnamon.}
		\end{minipage}
  \label{fig:linux-cinnamon}
\end{figure}
%\obr{linux-cinnamon}{Aplikace otevřená v~operačním systému Linux Mint 19 se správcem oken Cinnamon.}

\podsekce{Windows}
Pro Windows existuje alternativní instalační složka. Jejím stažením a rozzipováním do uživatelem zvolené složky vznikne spustitelná aplikace. 
%Jelikož na tomto operačním systému nejsou běžně dostupné potřebné dynamicky linkované knihovny Qt,
%instalační složka je přímo obsahuje.
%Důsledkem této skutečnosti je, že aplikace zabírá mnohem více diskového prostoru.
Je dobré vytvořit zástupce, přes kterého se bude daný program spouštět.
Při jeho vytváření se dá zvolit i pracovní adresář.
Jako ikonku je možné nastavit \c|Logo.ico|.
Zástupce je pak možné umístit například na plochu nebo do Start menu.
\obr{windows}{Aplikace otevřená v~operačním systému Windows 10.}

\podsekce{macOS}
Na tomto operačním systému je momentálně možná instalace pouze pomocí kompilace ze zdrojového kódu.


% TODO při inicializaci udělat ty složky.

\input{seznamStranekNaKontrolu.tex}
\sekce{Spuštění kontroly, tabulka změn, informační konzole}
Po spuštění aplikace se zobrazí hlavní okno.
Jelikož aplikace by měla běžet jako služba, okno neobsahuje tlačítko zavřít.
V~případě, že uživatel skutečné chce uzavřít aplikaci a tím i zabránit dalším automatickým kontrolám, je možné aplikaci ukončit z~menu kliknutím na \c|app| $\rightarrow$ \c|quit|.

Pro manuální spuštění kontroly je možné kliknout na tlačítko \c|Start checking|.
V~případě, že již kontrola běží, po kliknutí se ukončí a ihned začne znovu od začátku.
Po kliknutí na \c|Stop checking| se prohledávání neprodleně ukončí.

Při průchodu se nalezené změny ihned zobrazují do tabulky změn.
Tedy i v~případě, že je prohledávání ukončeno v~průběhu, doposud nalezené změny budou uloženy a zobrazeny.

Aktuální stav průchodu je vidět na stavovém baru  nad tlačítky.
V~případě, že bar má červenou barvu, alespoň jedna stránka nebyla při posledním průchodu úspěšně načtena.
Po kliknutí na zaškrtávací políčko \c|View more information| se zobrazí záznam o~provedených kontrolách a případně důvod jejich neúspěchu v~informační konzoli.
Všechny chybové hlášky jsou obsaženy na řádcích uvozených několika vykřičníky.
Nejčastěji se může uživatel setkat s~těmito chybami:\\
\c|Connection error - TIME OUT.| nastane v~případě, že načítání stránky bylo moc pomalé a tedy byl překročen časový limit na přístup na jednu stránku.\\ %isdo Opravit conection -> connection aplikaci
\c|Cookie error - cookie not available.| se zobrazí v~případě, že nastala chyba při načítání některé z~předcházejících stránek ze stejné cookie skupiny.\\
\c|Connection error - Network access is disabled.| znamená nedostupnost síťového připojení.\\
\c|Connection error - Host not found.| informuje o~nedostupnosti daného serveru. S~největší pravděpodobností se  jedná o~chybně zadanou stránku nebo byla přesunutá.
\obr{chyba-pristupu}{Výpis chyby v~informační konzoli.}

V~boxu vpravo od tlačítek je možno nastavit automatické spouštění kontrolování.
Do boxu stačí napsat počet minut mezi automatickými kontrolami.
Automatické kontroly lze vypnout napsáním \c|0| do políčka.
Datum a čas poslední úspěšné (tedy takové, ve které se načetly všechny stránky) kontroly je vidět mezi tlačítky a boxem automatické aktualizace.
V~tomto místě je také vidět čas příští plánované kontroly.

Informace o~poslední kontrole a periodě automatické kontroly si aplikace ukládá do souboru databáze v~pracovním adresáři aplikace.
Při spuštění si tento soubor načte (pokud existuje).
Nastavení automatické kontroly tedy vydrží i vypnutí a zapnutí aplikace.
V~případě, že kontrola měla proběhnout v~momentě, kdy byla aplikace vypnutá, proběhne neprodleně po jejím spuštění.

Když aplikace zaregistruje změnu stránky a není momentálně aktivní, bude se na změnu snažit upozornit.
Provedení této funkce je částečně závislé na platformě. 
Aplikace pošle požadavek na vyskočení do popředí, což ve většině případů znamená červené rozblikání dané aplikace případně ikony dané pracovní plochy.

Všechny nalezené změny se zobrazují v~tabulce změn umístěné v~horní části hlavního okna aplikace.
O~každé změně se zobrazí řádek obsahující jméno stránky, čas detekování změny a jméno souboru, v~němž je uložena aktuální verze.
V~případě, že uživatel chce data setřídit podle některé z~těchto položek, může tak učinit kliknutím na hlavičku daného sloupce tabulky.
Kliknutím na jméno souboru se ve výchozím prohlížeči otevře daná verze stránky.
Aby bylo zajištěno správné fungovaní zobrazení, jsou všechny relativní odkazy (v~rámci serveru) přepsány na absolutní.
Před všechny odkazy v~atributech \c|href| a \c|src|, které neobsahují absolutní cestu, je tedy doplněn název serveru a složky.
Díky tomu se při kontrole načítá pouze samostatná stránka, ale při zobrazení se načtou i obrázky, styly a další odkazované elementy.

V~případě, že už si uživatel danou změnu prohlédl, kliknutím na buňku v~sloupci \c|Delete| ji může z~tabulky změn odstranit.
Tímto odstraněním nedojde k~smazání záznamu o~změně ani odstranění souboru s~danou verzí stránky.
Změna se již nebude zobrazovat v~tabulce změn.
Jiným způsobem odstranění je vybrat jeden nebo několik řádků a pak kliknout na tlačítko \c|Hide| těsně pod tabulkou.

Vybráním řádků a kliknutím na tlačítko \c|Delete| dojde k~smazání vybraných záznamů změn včetně souborů obsahujících dané verze stránek.

Všechny záznamy o~změnách se ukládají do databáze a při spuštění aplikace se načtou všechny neskryté záznamy. 

\input{historie.tex}
\input{grafickePorovnavaniVerzi.tex}
%\sekce{Klávesové zkratky}

\kapitola{Implementace aplikace}

\sekce{Použitý software}
\podsekce{Qt}
Qt je multiplatformní knihovna pro c++ umožnující nejen tvorbu grafických aplikací, ale i práci se soubory, obrázky či sítovým připojením.
Byla nedílnou součástí vyvíjené aplikace.
Knihovna podporuje běh samotného jádra aplikace -- načítání webových souborů a také se stará o~grafické zobrazování dat.
Její funkce jsou tedy použité při většině výpočtů aplikace.
\podsekce{Qt Creator}
Qt Creator je vývojové prostředí určené primárně pro vývoj aplikací využívajících knihovny Qt.
Programátorovi nabízí zvýrazňování syntaxe a doplňování názvů.
Umožňuje také kompilování, spouštění a debugování aplikace pomocí jednoduše ovladatelných nástrojů.
Dále také umožňuje ovládání gitu pomocí zabudovaných nástrojů.
V~tomto prostředí vznikla většina zdrojových kódů aplikace.
\obr{qtcreator}{Vývojové prostředí Qt Creator}
\podsekce{Git}
Git je systém na správu verzí využitý při vývoji této aplikace.
Umožňuje oddělení jednotlivých úkonů při vývoji, jejich zdokumentování a možnost zobrazení jejich historie.
Dále nabízí pomocí vzdálených repositářů jednoduchý způsob, jak aplikaci nahrát na web a zpřístupnit uživatelům.

\sekce{Objektový model, rozdělení problému}
	Samotná knihovna Qt je psaná objektově, a proto je vhodné navázat v~objektové orientovaném programování i při vývoji.
	Program jsem tedy rozčlenil do několika vzájemně provázaných tříd.
	Pro každou třídu existují dva soubory obsahující hlavičky a samotný program.

	Program jsem dále rozčlenil na téměř samostatné části.
	Třídy pozadí se starají o~samotné načítání a zjišťování rozdílů na stránkách, práci s~databází a porovnávání verzí stránek.
	Ostatní třídy se starají o~správné předávání dat uživateli pomocí graficky vykreslených oken a také přijímají pokyny od uživatele a propagují je do příslušných tříd.
	Díky tomu je možné poměrně jednoduše přepsat celé uživatelské rozhraní například do podoby konzolové aplikace s~využitím stávájících funkcí pozadí.


\sekce{Pozadí aplikace}
Hlavní třídou aplikace je \c|Background|. Ta se stará o~inicializaci a propojení ostatních tříd pozadí.
Dále si také ukládá a načítá seznam stránek na kontrolu.

Třída \c|ConnectionThread| se stará o~samotné načítání a hledání rozdílů ve stránkách. Aby byl zajištěn hladký chod aplikace, je tato část programu vykonávána jako speciální vlákno. % TODO přejmenovat to
Díky tomu lze při načítání libovolně pracovat s~aplikací.
V~této třídě je také implementováno ignorování částí změn na základě zadaných podmínek.

Pro usnadnění přístupu do databáze slouží třída \c|Database|.
Ta přistupuje do souboru \c|database.db| % TODO přejmenovat to.
pomocí knihovny SQLite.
V~databázi jsou uloženy záznamy o~všech verzích stránek a také uživatelské nastavení.

O~ukládání nových změn na stránce se stará \c|NewPageModel|. Tento tabulkový model je pak přímo napojen na zobrazování změn.

Třída \c|PageQuery| založená na \c|NewPageModel| obsahuje tabulky dotazu z~databáze na historii změn.

\podsekce{Porovnávání verzí stránek}
\c|PageComparator| umožňuje porovnávání verzí stránek.
Jeho úkolem je najít nejlepší napojení stránek na sebe a následné vytvoření souborů, které zobrazují změny v~prohlížeči.

Obě dvě verze stránky si rozdělí na jednotlivá slova a tagy. S~nimi se dále pracuje jako se znaky.
Poté hledá nejdelší shodný podřetězec.
Na to využívá dynamického programování.
Pro každou dvojici prefixů z~obou řetězců si spočítá délku nejdelšího shodného podřetězce.
K~výpočtu ovšem využívá toho, že tuto hodnotu již zná pro kratší prefixy.
Když je poslední slovo v~obou prefixech stejné, tak oproti dvojici, v~niž jsou oba prefixy o~jedno slovo kratší, může být délka podřetězce o~jedna delší (toto odpovídá variantě, že poslední slovo přibude ve shodném podřetězci).
Délka podřetězce ovšem může být stejná jako délka podřetězce ve dvojicích, které mají jeden sufix o~jedna kratší (toto odpovídá variantě, že shodný podřetězec zůstane stejný).
Z~těchto variant vybere maximum.

Poté již není problém zrekonstruovat tvar podřetězce.
Stačí projít takto vytvořenou tabulku od prefixů obsahujících celé řetězce směrem odkud vzniklo maximum v~dané buňce.

Při tomto průchodu si poznačí, které části vstupu jsou součástí shodného podřetězce a které nikoliv.
Části, které nejsou součástí podřetězce pak zobrazuje jako změny, tedy je barevně zvýrazní obalením do tagu \c|span|.




\sekce{Grafické uživatelské rozhraní}
Hlavní okno aplikace je implementováno pomocí třídy \c|MainWindow|. Toto je hlavní třída celého projektu.
Z~ní se teprve spouští inicializace pozadí aplikace včetně \c|Background|.

O~správné vykreslování tabulek změn stránek včetně napojení jejich ovládacích prvků se stará \c|PageViewer|.
\c|PageComparatorGUI| se stará o~okno zobrazené při porovnávání stránek a jeho napojení na \c|PageComparator|.
Okno historie změn zobrazuje \c|HistoryWindow|.


\kapitola{Závěr}

Podařilo se mi vytvořit funkční aplikaci, kterou momentálně šířím mezi první uživatele.
Sám aplikaci již delší dobu používám a také testuji její funkčnost na různých zařízeních či operačních systémech.
Pravidelně opravuji chyby a implementuji doporučení od uživatelů.
Dále bych aplikaci rád rozvíjel přidáním dalších funkcí či rozšířením stávajících.


\kapitola{Resume}
Many people sometimes need to check change on web pages, such as when they wait for publishing some information or document.
I wanted to help  them and save their time  needed for manual checking some pages repeatedly.
I developed a desktop application, that can do this automatically.
The name of this application is Web Updating  Indicator.

I used the Qt library in c++ language which implements network connection, computing engine and graphic user interference.
I developed this app in Qt Creator, which is large Qt IDE with most functions such as code completing and hinting for c++ and Qt functions.
Git was used for version control of all project.
Everything (code and binary files) was publish thought them to Gitlab.
This app come under GNU LGPL licence.

I succeeded to deploy the application.
It is running on most of using Linux distribution and Windows as well.
A few people are using and testing this now.
It can check complex web pages with smart tools for comparing pages and log in to web app by cookie session. Different pages are shown in main window of this application.
User can view changes between two versions of page by graphic tools which open the page with colour marked changes.
It is possible to display all detected version of page by SQL query to database directly from the app whit graphic output.




\input{literatura.tex}

\end{document}
