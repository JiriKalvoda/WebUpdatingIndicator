\pdfoutput=1
\documentclass[twoside,12pt]{article}%
\usepackage[utf8]{inputenc} % použito v případě jiného kódování
% aktuální kódování: utf8
\usepackage{dipp}
\usepackage[czech]{babel}
\usepackage{graphicx}


% ~~~~~ PACKAGE ~~~~~
\usepackage[utf8]{inputenc}  % základní kódování
\usepackage[czech]{babel}    % české nastavení
\usepackage{amsmath, amssymb}   % matematické simboly
% amsfonts
\usepackage{graphicx}  % barvy a jiné
\usepackage{lastpage}   % zjištění počtu stránek
\usepackage{calc}  % snazší výpočty délek
\usepackage{refcount} % převod reference na číslo - dvojstraný režim - dělení 2
\usepackage{fancyhdr} % hlavičky
\usepackage{etoolbox}  % změna geimetrie na určité stránky
\usepackage{pgfplots} % geogebra
\usepackage[final]{pdfpages} % geogebra v pdf
%mathrsfs, pdf, tikz, mathrsfs
\usepackage{listingsutf8} % zdrojové kódy
\usepackage{hyperref}
% ~~~~~ MATEMATICKÉ ~~~~~
\renewcommand{\(}{\left(}
\renewcommand{\)}{\right)}
\renewcommand{\[}{\left[}
\renewcommand{\]}{\right]}
\def\MATHINFO{}
  \newcommand{\mathinfo}[2]{\if ^\MATHINFO^ \underbrace{\def\MATHINFO{1} {#1} \def\MATHINFO{} }_{\text{#2}} \else \overbrace{\def\MATHINFO{} {#1} \def\MATHINFO{1}}^{\text{#2}} \fi}
\newcommand{\Mathinfo}[1]{\qquad \qquad \text{({#1})}}

\newcommand{\tg}{\mathrm{tg}}
\newcommand{\cotg}{\mathrm{cotg}}
\newcommand{\sgn}{\mathrm{sgn}}
\renewcommand{\angle}{\sphericalangle}
\newcommand{\degre}{\ensuremath{^\circ}}
\def\d{\degre}
\newcommand{\pri}{\overleftrightarrow}
\newcommand{\ppri}{\overrightarrow}
\def\imp{\Rightarrow}
\def\Imp{\quad\imp\quad}
\def\ekv{\Leftrightarrow}
\def\Ekv{\quad\ekv\quad}
\def\rimp{\Leftarrow}
\def\RImp{\quad\rimp\quad}
\renewcommand{\*}{\cdot{}}
\def\N{\mathbb{N}}
\def\Z{\mathbb{Z}}
\def\Q{\mathbb{Q}}
\def\R{\mathbb{R}}
\def\C{\mathbb{C}}
\def\E{\mathbb{E}}
\def\P{\mathbb{P}}
\def\US{\leftrightarrow{}}

\newcommand{\eqnl}[1][]{{#1}$$\\[-16px]$${}{#1}}

\newcommand{\derivation}[2]{\frac{\partial}{\partial {#1}}\({#2}\)}
\renewcommand{\j}[2][]{\ \mathrm{\if ^#1^ {#2} \else \frac{{#1}}{{#2}} \fi}}

\def\f{\frac}
\def\uv#1{„#1“}

\newcounter{lemmaCounter}[section]
\renewcommand{\thelemmaCounter}{\arabic{lemmaCounter}}
\newtheorem{lemmaBlock}[lemmaCounter]{}
\newenvironment{lemma}[3][]
{

	\refstepcounter{lemmaCounter}
		\label{#2}
	\textbf{Lemma \thelemmaCounter\if^#1^\else \ ({#1})\fi:}
	\emph{{#3}}\\{}
	Důkaz: 
}
{ 
\hfill $\Box$

}

% ~~~~~ INFORMATICKÉ ~~~~~
\newcommand{\refcodeline}[1]{{\footnotesize Viz \ref{#1}. řádek kódu na straně \pageref{#1}.}}
\newcommand{\refcodeblock}[1]{{\footnotesize Viz \ref{#1}.-\ref{#1-end}. řádek kódu na straně \pageref{#1}.}}
\definecolor {lightGrey}{RGB}{250,250,250}
\newcommand{\Onotation}[1]{\ifmmode\mathcal{O}({#1})\else$\mathcal{O}({#1})$\fi}
\renewcommand{\O}{\Onotation}
\newcommand{\codeF}[1]{\lstinputlisting[ numbers = left, numberstyle = \tiny ,frame = shadowbox, backgroundcolor =\color{lightGrey}, showstringspaces=false]{#1} }
\newcommand{\codeMain}{\codeF{main.cpp}}
\lstset{language = C++,basicstyle =  \small \ttfamily, keywordstyle = \emph,commentstyle = \rmfamily ,  backgroundcolor =\color{lightGrey}, showstringspaces=false, escapeinside ={///}{/}, extendedchars=true, 
literate={á}{{\'a}}1 {í}{{\'i}}1 {é}{{\'e}}1 {ý}{{\'y}}1 {ú}{{\'u}}1 {ó}{{\'o}}1 {ě}{{\v{e}}}1 {š}{{\v{s}}}1 {č}{{\v{c}}}1 {ř}{{\v{r}}}1 {ž}{{\v{z}}}1 {ď}{{\v{d}}}1 {ť}{{\v{t}}}1 {ň}{{\v{n}}}1 {ů}{{\r{u}}}1 {Á}{{\'A}}1 {Í}{{\'I}}1 {É}{{\'E}}1 {Ý}{{\'Y}}1 {Ú}{{\'U}}1 {Ó}{{\'O}}1 {Ě}{{\v{E}}}1 {Š}{{\v{S}}}1 {Č}{{\v{C}}}1 {Ř}{{\v{R}}}1 {Ž}{{\v{Z}}}1 {Ď}{{\v{D}}}1 {Ť}{{\v{T}}}1 {Ň}{{\v{N}}}1 {Ů}{{\r{U}}}1  ,}
\def\c{\lstinline}




